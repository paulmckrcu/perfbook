% appendix/rcuhist/RCUinLinux.tex
% SPDX-License-Identifier: CC-BY-SA-3.0

\chapter{Read-Copy Update in Linux}
\label{app:rcuhist:Read-Copy Update in Linux}

This chapter gives a history of RCU in the Linux kernel from mid-2008
onwards.
Earlier history of RCU may be found
elsewhere~\cite{PaulEdwardMcKenneyPhD,PaulEMcKenney2008RCUOSR}.
Section~\ref{sec:app:rcuhist:RCU Usage Within Linux}
gives an overview of the growth of RCU usage in Linux and
Section~\ref{sec:app:rcuhist:RCU Evolution}
presents a detailed view of recent RCU evolution.

\section{RCU Usage Within Linux}
\label{sec:app:rcuhist:RCU Usage Within Linux}

\begin{figure}[bp]
\begin{center}
\resizebox{3in}{!}{\includegraphics{appendix/rcuhist/linux-RCU}}
\end{center}
\caption{RCU API Usage in the Linux Kernel}
\label{fig:app:rcuhist:RCU API Usage in the Linux Kernel}
\end{figure}

\begin{figure}[bp]
\begin{center}
\resizebox{3in}{!}{\includegraphics{appendix/rcuhist/linux-RCUlock}}
\end{center}
\caption{RCU API Usage in the Linux Kernel vs. Locking}
\label{fig:app:rcuhist:RCU API Usage in the Linux Kernel vs. Locking}
\end{figure}

\begin{figure}[tbp]
\begin{center}
\resizebox{3in}{!}{\includegraphics{appendix/rcuhist/rcuAPI}}
\end{center}
\caption{RCU API Growth Over Time}
\label{fig:app:rcuhist:RCU API Growth Over Time}
\end{figure}

The Linux kernel's usage of RCU has increased over the years,
as can be seen from
Figure~\ref{fig:app:rcuhist:RCU API Usage in the Linux Kernel}~\cite{PaulEMcKenneyRCUusagePage}.
RCU has replaced other synchronization mechanisms
in existing code
(for example, \co{brlock} in the networking protocol
stacks~\cite{Molnar00a,Torvalds2.5.69,Torvalds2.5.70}),
and it has also been introduced with code implementing
new functionality
(for example, the audit system within SELinux~\cite{JamesMorris04b}).
However, RCU remains a niche technology compared to locking,
as shown in
Figure~\ref{fig:app:rcuhist:RCU API Usage in the Linux Kernel vs. Locking}.
If locking is the hammer in the kernel hacker's concurrency toolbox,
perhaps RCU is the screwdriver.
If so, it is an rapidly evolving screwdriver, as can be seen in
Figure~\ref{fig:app:rcuhist:RCU API Growth Over Time}.

\section{RCU Evolution}
\label{sec:app:rcuhist:RCU Evolution}

This section presents ongoing experience with RCU since mid-2008.

\subsection{2.6.27 Linux Kernel}

This release added the
\co{call_rcu_sched()},
\co{rcu_barrier_sched()}, and
\co{rcu_barrier_bh()} RCU API members.

\subsection{2.6.28 Linux Kernel}

One welcome change involved an actual reduction in the size of RCU's
API with the removal of the \co{list_for_each_rcu()} primitive.
This primitive is superseded by \co{list_for_each_entry_rcu()},
which has the advantage of iterating over structures rather than
iterating over the pointer pairs making up a \co{list_head}
structure (which, confusingly, acts as a list element as well
as a list header).
This change was accepted into the 2.6.28 Linux kernel.

Unfortunately, the 2.6.28 Linux kernel also added
\co{rcu_read_lock_sched()} and
\co{rcu_read_unlock_sched()} RCU API members.
These APIs were added to promote readability.
In the past, primitives to disable interrupts or preemption were used
to mark the RCU read-side critical sections corresponding to
\co{synchronize_sched()}.
However, this practice led to bugs when developers removed the need
to disable preemption or interrupts, but failed to notice the need
for RCU protection.
Use of \co{rcu_read_lock_sched()} will help prevent such bugs in the
future.

\subsection{2.6.29 Linux Kernel}

A new more-scalable implementation, dubbed ``Tree RCU'', replaces
the flat bitmap with a combining tree, and was accepted into the 2.6.29
Linux kernel.
This implementation was inspired by the ever-growing core counts of
modern multiprocessors, and is designed for many hundreds of CPUs.
Its current architectural limit is 262,144 CPUs, which the developer
(perhaps na\"ively) believes to be sufficient for quite some time.
This implementation also adopts preemptible RCU's improved dynamic-tick
interface.

Mathieu Desnoyers added
\co{rcu_read_lock_sched_notrace()} and
\co{rcu_read_unlock_sched_notrace()},
which are required to permit the tracing code in the Linux kernel
to use RCU.
Without these APIs, attempts to trace RCU read-side critical sections
lead to infinite recursion.

Eric Dumazet added a new type of RCU-protected list that allows single-bit
markers to be stored in the list pointers.
This type of list enables a number of lockless algorithms, including
some reported on by Maged Michael~\cite{MagedMichael04a}.
Eric's work adds the \co{hlist_nulls_add_head_rcu()},
\co{hlist_nulls_del_rcu()}, \co{hlist_nulls_del_init_rcu()}, and
\co{hlist_nulls_for_each_entry_rcu()}.
It also adds a new structure named \co{hlist_nulls_node}.

Although it is strictly speaking not part of the Linux kernel, 
at about this same time, Mathieu Desnoyers announced his user-space
RCU implementation~\cite{MathieuDesnoyers2009URCU}.
This is an important first step towards a real-time user-level RCU
implementation.

\subsection{2.6.31 Linux Kernel}

Jiri Pirko added \co{list_entry_rcu} and \co{list_first_entry_rcu()}
primitives that encapsulate the \co{rcu_dereference()} RCU-subscription
primitive into higher-level list-access primitives, which will hopefully
eliminate a class of bugs.

In addition, the ``Tree RCU'' implementation was upgraded from
``experimental'' status.

\subsection{2.6.32 Linux Kernel}

Perhaps the largest change in this version of the Linux kernel
is the removal of the old ``Classic RCU'' implementation.
This implementation is superseded by the ``Tree RCU'' implementation.

This version saw a number of other changes, including:

\begin{enumerate}
\item	The appearance of \co{synchronize_rcu_expedited()},
	\co{synchronize_sched_expedited()}, and
	\co{synchronize_rcu_bh_expedited()} RCU API members.
	These primitives are equivalent to their non-expedited
	counterparts, except that they take measures to expedite the
	grace period.
\item	Add preemptible-RCU functionality to the ``Tree RCU''
	implementation, thus removing one obstacle to real-time
	response from large multiprocessor machines running Linux.
\item	This new ``Tree Preemptible RCU'' implementation obsoletes
	the old preemptible RCU implementation, which was removed
	from the Linux kernel.
\end{enumerate}

\subsection{2.6.33 Linux Kernel}

Perhaps the most dramatic addition to this release was
a day-one bug in Tree RCU~\cite{PaulEMcKenney2009HuntingHeisenbugs}.
Other changes include:

\begin{enumerate}
\item	``Tiny RCU'', also known as ``RCU: The Bloatwatch
	Edition''~\cite{PaulEMcKenney2009LWNBloatWatchRCU}.
\item	Expedited SRCU in the form of
	\co{synchronize_srcu_expedited()}.
\item	A cleanup of Tree RCU synchronization prompted by the
	afore-mentioned bug.
\item	Add expedited implementation for Tree Preemptible RCU
	(in earlier releases, ``expedited'' support had simply
	mapped to \co{synchronize_rcu()}, which is semantically
	correct if somewhat unhelpful from a performance viewpoint.)
\item	Add a fourth level to Tree RCU, which improves stress testing.
	Therefore, if someone ever wants to run Linux on a system with
	16,777,216 CPUs, RCU is ready for them!
	Give or take the response-time implications of scanning
	through 16 million per-CPU data elements...
\end{enumerate}

\subsection{2.6.34 Linux Kernel}

The most visible addition for this release was \co{CONFIG_PROVE_RCU},
which allows \co{rcu_dereference()} to check for correct locking
conditions~\cite{PaulEMcKenney2010LockdepRCU}.
Other changes include:

\begin{enumerate}
\item	Simplifying Tree RCU's interactions between
	forcing an old grace period and starting a new one.
\item	Rework counters so that free-running counters are unsigned.
	(You simply cannot imagine the glee on the faces of certain
	C-compiler hackers while they discussed optimizations that would
	break code that naively overflowed signed integers!!!)
\item	Update Tree Preemptible RCU's stall detection to print out
	any tasks preempted for excessive time periods while in
	an RCU read-side critical section.
\item	Other bug fixes and improvements to Tree RCU's CPU-stall-detection
	code.
	This code checks for CPUs being locked up, for example,
	in infinite loops with interrupts disabled.
\item	Prototype some code to accelerate grace periods when the
	last CPU goes idle in battery-powered multiprocessor
	systems.
	There were people who were quite unhappy about RCU taking
	a few extra milliseconds to get the system in a state
	where all CPUs could be powered down!
\end{enumerate}

\subsection{2.6.35 Linux Kernel}

This release includes a number of bug fixes and cleanups.
The major change is the first installment of Mathieu Desnoyers's
patch to check for misuse of RCU callbacks, for example, passing
a \co{rcu_head} structure to \co{call_rcu()} a second time within
a single grace period.

\subsection{2.6.36 Linux Kernel}

The core of Mathieu Desnoyers's debugobjects work appeared in 2.6.36,
with some cleanups deferred to 2.6.37 due to dependencies on commits
flowing up other maintainer trees.
A key piece of Arnd Bergmann's sparse RCU checking appeared in 2.6.36,
with the remainder deferred to 2.6.37, again due to dependencies on
commits flowing up other maintainer trees.
Finally, a patch from Eric Dumazet fixed an error in
\co{rcu_dereference_bh()}'s error checking.

\subsection{2.6.37 Linux Kernel}

The final cleanups from Mathieu Desnoyers's debugobjects work appeared
in 2.6.37, as did the remainder of Arnd Bergmann's sparse-based checking work.
Lai Jiangshan added some preemption nastiness to rcutorture and
made some simplifications to Tree RCU's handling of per-CPU data.
Tetsuo Handa fixed an RCU lockdep splat, Christian Dietrich removed
a redundant \co{#ifdef}, and Dongdong Deng added an
\co{ACCESS_ONCE()} that help call out lockless accesses to some
Tree RCU control data.

Paul's implementation of preemptible Tiny RCU also appeared in
2.6.37, as did a number of enhancements to the RCU CPU stall-warning
code, docbook fixes, coalescing of duplicate code, Tree RCU speedups,
added tracing to support queuing models on RCU callback flow,
and several miscellaneous fixes and cleanups.

\subsection{2.6.38 Linux Kernel}

Lai Jiangshan moved \co{synchronize_sched_expedited()} out of
\co{kernel/sched.c} and into \co{kernel/rcutree.c} and
\co{kernel/rcu_tiny.c} where it belongs.
He also simplified RCU-callback handling during CPU-hotplug operations
by eliminating the \co{orphan_cbs_list}, so that RCU callbacks
orphaned by a CPU that is going offline are immediately adopted by
the CPU that is orchestrating the offlining sequence.
Tejun Heo improved \co{synchronize_sched_expedited()}'s batching
capabilities, which in turn improves performance and scalability
for workloads with many concurrent \co{synchronize_sched_expedited}
operations.
Fr\'ed\'eric Weisbecker provided a couple of subtle changes to the
RCU core code that make RCU more power-efficient when idle.
Mariusz Kozlowski fixed an embarrassing syntax error in
\co{__list_for_each_rcu()}, which was then removed.
(But the fixed version is there in the git tree should it be needed.)
Nick Piggin added the \co{hlist_bl_set_first_rcu()},
\co{hlist_bl_first_rcu()},
\co{hlist_bl_del_init_rcu()},
\co{hlist_bl_del_rcu()},
\co{hlist_bl_add_head_rcu()}, and
\co{hlist_bl_for_each_entry_rcu()} primitives for RCU-protected use
of bit-locked doubly-linked lists.
Christoph Lameter implemented \co{__this_cpu_read()}, which is an
optimized variant of
\co{__get_cpu_var()} for use in cases where the variable is accessed directly.

In addition, \co{TINY_RCU} gained priority boosting, a race condition
in \co{synchronize_sched_expedited()} was fixed,
\co{synchronize_srcu_expedited()} was modified to retain its expedited
nature in the face of concurrent readers,
grace-period begin/end checks were improved,
and the \co{TREE_RCU} leaf-level fanout was limited to 16 in order to fix
lock-contention problems.
This last change reduces the maximum number of CPUs that \co{TREE_RCU}
and \co{TREE_PREEMPT_RCU} can support down to 4,194,304, which is
(again, perhaps na\"ively) believed to be sufficient.

\subsection{2.6.39 Linux Kernel}

Lai Jiangshan made \co{TINY_RCU}'s \co{exit_rcu()} invoke
\co{__rcu_read_unlock()}
rather than \co{rcu_read_unlock()} in case of a task exiting while
in an RCU read-side critical section in order to preserve debugging
state,
Jesper Juhl removed a duplicate include of \co{sched.h} from
rcutorture,
and
Amerigo Wang removed some dead code from \co{rcu_fixup_free()}.

In addition, a new \co{rcu_access_index()} was created for use in the
MCE subsystem.

\subsection{3.0 Linux Kernel}

What many expected to be the 2.6.40 release became instead the 3.0 release.
The most important RCU feature was the addition of priority boosting
for Tree RCU: Important in more ways than
planned~\cite{PaulEMcKenney2011RCU3.0trainwreck}, resulting in RCU
fixes after 3.0-rc7.
Kudos to Shaohua Li, Peter Zijlstra, Steven Rostedt for much help
dealing with the fallout of the collision between RCU, the scheduler,
and threaded interrupts.
In addition, RCU CPU stall warnings are now unconditionally compiled
into Tree RCU, though they may still be disabled via the
\co{rcu_cpu_stall_suppress} module parameter, which may be controlled
from either the kernel boot parameter string or \co{sysfs}.

Mathieu Desnoyers enabled \co{DEBUG_OBJECTS_RCU_HEAD} checking to
be carried out in non-preemptible RCU implementations.
Lai Jiangshan created a fire-and-forget \co{kfree_rcu()} (and applied
it throughout the kernel),
and also made \co{TREE_RCU}'s \co{exit_rcu()} invoke
\co{__rcu_read_unlock()}
rather than \co{rcu_read_unlock()} in case of a task exiting while
in an RCU read-side critical section in order to preserve debugging
state.
Eric Dumazet further shrank \co{TINY_RCU} and
Gleb Natapov added RCU hooks to allow virtualization to call RCU's
attention to quiescent states that occur when switching context to
and from a guest OS.
Peter Zijlstra streamlined RCU kthread blocking and wakeup.

\subsection{3.1 Linux Kernel}

The 3.1 version was a quiet time for RCU, with cleanups and minor fixes
from Arun Sharma, Jiri Kosina, Michal Hocko, Peter Zijlstra,
and Jan H. Sch\"{o}nherr.

\subsection{3.2 Linux Kernel}

The 3.2 Linux kernel contains a number of fixes to issues located
during the first phase of a top-to-bottom inspection of RCU's code.
One outcome of this inspection is that deadlock can occur if
an irq-disabled section of code overlaps the end but not the beginning
of a preemptible RCU read-side critical section.
Therefore, do not code RCU read-side critical sections that partially
overlap with irq-disabled code sections:
Instead, either fully enclose the irq-disable code sections within a
given RCU read-side critical section or vice versa.

This release saw the first RCU event-tracing capabilities.
Eric Dumazet applied the new
\co{kthread_create_on_node()} primitive to ensure that RCU's
kthreads have memory placed optimally on NUMA systems.
He also
made the \co{rcu_assign_pointer()} unconditionally insert a
memory barrier because the earlier compiler magic permitting this
barrier to be omitted under certain circumstances fails in newer
versions of the compiler.
Therefore, when assigning \co{NULL} to an RCU-protected pointer,
use \co{RCU_INIT_POINTER()} rather than \co{rcu_assign_pointer()}.

Shaohua Li eliminated an unnecessary self-wakeup of RCU's per-CPU
kthreads, and Andi Kleen cleaned up some conflicting variable
declarations.
Mike Galbraith fixed a bug that caused RCU to ignore the
\co{RCU_BOOST_PRIO} kernel parameter, and finally,
rcutorture made some headway in catching up to the ever-expanding
RCU capabilities.

\subsection{3.3 Linux Kernel}

The 3.3 Linux kernel contains
energy-efficiency improvements that reduce RCU's need for
scheduling-clock ticks from otherwise idle CPUs,
a new \co{srcu_read_lock_raw()} primitive needed by uprobes,
additional fixes for issues located in the still-ongoing top-to-bottom
inspection of RCU,
and improvements to \co{rcutorture} that enable scripted KVM-based
testing of RCU, independent of the type or presence of userspace layout.

Also included were some \co{-rt} RCU patches from Thomas Gleixner,
as well as
a number of RCU-infrastructure patches from Fr\'ed\'eric Weisbecker
in support of the long-hoped-for application of dyntick-idle mode
to usermode execution.

Although some initial work has gone into permitting RCU-preempt's
\co{__rcu_read_lock()} and \co{__rcu_read_unlock()} to be inlined,
much more work is needed to disentangle various include-file issues.
Finally, there were miscellaneous fixes from Rusty Russell.

There has been an initial request for \co{rcu_barrier_expedited()},
but given that the requester found another way to solve this problem,
this has relatively low priority.

\subsection{3.4 Linux Kernel}

The 3.4 kernel contains yet more energy-efficiency work, reducing their
downsides to rapid idle entry/exit workloads.
The tradeoff managed here is increased work on idle entry compared to
longer idle times, and so the changes in this release do a better job of
recognizing when additional effort is futile, for example, if the
CPU is entering and exiting idle rapidly due to the workload, there is
little point in taking idle-entry actions that would allow the CPU to
stay asleep longer.

This release also added \co{RCU_NONIDLE()}, which is used to handle
the increasingly frequent practice of invoking RCU from idle CPUs.
Because RCU ignores idle CPUs, this practice is quite dangerous.
The new \co{RCU_NONIDLE()} macro therefore carries out a momentary
exit from idle so that RCU read-side critical sections can do their job.

RCU's handling of CPU hotplug was improved, rcutorture gained some
primitive ability to test RCU CPU stalls warnings, and the stall
warnings themselves were improved by adding more information and
by adding the ability to control timeouts via sysfs.
\co{TREE_RCU} no longer may be used in \co{CONFIG_SMP=n} kernels;
\co{TINY_RCU} is used instead.
This release also saw the addition of lockdep-RCU checks for sleeping
in a non-preemptible-RCU read-side critical section, as well as for
entering the idle loop while in an RCU read-side critical section.

\co{TINY_RCU} inherited the \co{TREE_RCU} fixes for the v3.0-rc7 RCU
trainwreck~\cite{PaulEMcKenney2011RCU3.0trainwreck}.
The grace-period initialization process dropped the old single-node
optimization, and callbacks remaining on offlined CPUs no longer
need to go through a second full grace period.
Furthermore, offline CPUs are no longer permitted to invoke RCU
callbacks.

Yet more tweaks to the energy-efficiency code limited the amount of
time lazy callbacks could languish on an idle CPU.
Finally, a number of fixes were supplied by Fr\'ed\'eric Weisbecker,
Heiko Carstens, Julia Lawall, Hugh Dickins, Jan Beulich, and Paul
Gortmaker.

\subsection{3.5 Linux Kernel}

The 3.5 Linux kernel included yet more adjustments to the
\co{CONFIG_RCU_FAST_NO_HZ} energy-efficiency code, including timer
handling and proper handling of \co{RCU_NONIDLE()} pauses out of
idle.

It also included work to reduce the disruption due to \co{rcu_barrier()}
and friends by avoiding enqueuing callbacks on CPUs that have none.
This work also made the interaction between \co{rcu_barrier()} and
callbacks orphaned by offlined CPUs more explicit, which was required
in order to avoid some nasty race conditions.
An abortive attempt to inline \co{__rcu_read_unlock()} left but one
commit that consolidated and reduced the overhead of RCU's
task-exit handling.

This release contains a complete rewrite of SRCU by Lai Jiangshan as
well as fixes from Jan Engelhardt, Michel Machado, and Dave Jones.

\subsection{3.6 Linux Kernel}

The 3.6 Linux kernel included the first round of changes to reduce
RCU's scheduling-latency impact on systems with thousands of CPUs,
namely allowing leaf-level fanout of the \co{rcu_node} tree to be
controlled by a boot-time parameter.
This change reduced the amount of memory that needed to be touched
during grace-period initialization by a factor of four, thus reducing
the latency impact from about 200 microseconds to 60-70 microseconds.
This release also increased \co{rcu_barrier()} concurrency.

Following an established tradition, this release also contained
energy-efficiency changes for the \co{CONFIG_RCU_FAST_NO_HZ}
facility.
Finally, the release contained a number of fixes, including
an uninitialized-string fix from Carsten Emde.

\subsection{3.7 Linux Kernel}

The 3.7 Linux kernel moved grace-period initialization to a separate
kthread, where it is preemptible, which should eliminate
grace-period-initialization-latency problems on large systems.
This release also removed the previous \co{_rcu_barrier()} dependency
on the much-maligned \co{__stop_machine()}.
It also contained some of the RCU infrastructure required by
Fr\'ed\'eric Weisbecker's \co{CONFIG_NO_HZ_FULL} bare-metal
facility~\cite{JonCorbet2013NO-HZ-FULL}, and much of this RCU
infrastructure was in fact also written by Fr\'ed\'eric.
Finally, it contained fixes and optimizations from Tejun Heo,
Thomas Gleixner, Li Zhong, and Dimiti Sivanich.

\subsection{3.8 Linux Kernel}

The 3.8 Linux kernel added a prototype implementation of RCU callback
offloading in the form of a new \co{CONFIG_RCU_NOCB_CPU} Kconfig
parameter~\cite{JonCorbet2012NOCB}, for which Paul Gortmaker provided
a couple of badly needed fixes.
This prototype implementation requires that CPU~0 not be offloaded,
and in fact that all callbacks be handled by CPU~0.
This is clearly not scalable, so a better implementation will appear later.
RCU CPU stall-warning messages were once again upgraded, and some
improvements to RCU's CPU-hotplug code were added.

Lai Jiangshan added static definition capability to SRCU and Michael
Wang reworked RCU's old debugfs tracing facility.
Antti P.~Miettinen added a kernel boot parameter that forces all RCU
synchronous grace-period primitives to execute in expedited mode,
and Eric Dumazet fixed an RCU callback batch-limit problem.

\subsection{3.9 Linux Kernel}

The 3.9 Linux kernel tags groups of callbacks with the corresponding
number, which allows RCU to be maximally aggressive about promoting
callbacks with no need to worry about over-promoting them.
In addition, this release adds RCU CPU stall warnings for \co{TINY_RCU}.

Lai Jiangshan provided some SRCU updates, allowing SRCU read-side primitives
to be invoked from idle and offline CPUs, along with some additional
fixes.
Additional fixes were provided by Sasha Levin, Steven Rostedt,
Li Zhong, Cody P. Schafer, and Josh Triplett.

\subsection{3.10 Linux Kernel}

With the 3.10 Linux kernel, RCU finally has an energy-efficiency mechanism
that delivers energy savings that are measurable at the system
level~\cite{PaulMcKenney2013AMPenergyHOTPAR}.
The trick is making \co{CONFIG_RCU_FAST_NO_HZ} use the callback-tagging from
3.9.
This means that CPUs going idle need only classify and number their own
callbacks, which is considerably cheaper than than the prior approach
of attempting to force the RCU state machine forward.
In addition, the callback-tagging was enhanced to allow CPUs to indicate
the need for future grace periods, which allows CPUs to indicate a need
for a grace period, and to have that grace period complete, despite the
fact that the requesting CPU was asleep through the whole process.

In addition, the \co{CONFIG_RCU_NOCB_CPU} facility was improved to
remove its dependency on CPU~0, thus allowing RCU callbacks to be offloaded
from all CPUs.

This release also included fixes from Steven Rostedt, Eric Dumazet,
Sasha Levin, Fr\'ed\'eric Weisbecker, Al Viro, Steven Whitehouse,
Srivatsa S.~Bhat, Jiang Fang, and Akinobu Mita.

\subsection{3.11 Linux Kernel}

The 3.11 Linux kernel added cleanups for the callback-tagging work
in 3.9 and 3.10 and removed \co{TINY_PREEMPT_RCU} in favor of running
\co{TREE_PREEMPT_RCU} in uniprocessor mode.
This release also includes fixes from Paul Gortmaker and Kees Cook.

\subsection{3.12 Linux Kernel}

The 3.12 kernel adds the \co{CONFIG_NO_HZ_FULL_SYSIDLE} Kconfig
parameter that provides the infrastructure required to allow
\co{CONFIG_NO_HZ_FULL} to efficiently determine when the entire
system is idle.
This is important because unless \co{CONFIG_NO_HZ_FULL} can prove
that the full system is idle, it must force CPU~0 to keep its
scheduling-clock interrupt active, which is not so good for battery
lifetime~\cite{JonathanCorbet2013SYSIDLE}.

This release also improved rcutorture's test coverage by testing
synchronous, asynchronous, and expedited grace-period primitives
in parallel.
It also adds duplicate-callback testing and makes rcutorture give
more information when a CPU-online operation fails.
Finally, it includes fixes from Steven Rostedt, Tejun Heo, and Borislav
Petkov.

\subsection{3.13 Linux Kernel}

The 3.13 kernel contains some improvements in \co{CONFIG_RCU_FAST_NO_HZ}
execution, especially avoiding too-frequent attempts to advance callbacks.
The rationale is that those events permitting callbacks to advance
typically occur only every few milliseconds, so attempting to advance callbacks
more frequently than once per jiffy does nothing but reduce performance and
waste power.
The 3.13 kernel therefore does not attempt to advance callbacks if it
has already done so within the current jiffy.

A new \co{rcu_is_watching()} function allows the caller to determine
whether or not it is safe to enter an RCU read-side critical section.
In other words, \co{rcu_is_watching()} returns true unless the CPU is
either idle or offline.
In addition, a new \co{smp_mb__after_srcu_read_unlock()}
interface (provided by Michael S.~Tsirkin)
guarantees a full memory barrier from \co{srcu_read_unlock()}.
Note that although \co{srcu_read_unlock()} currently already provides
a full memory barrier, earlier implementations did not do so and
future implementations might once again not do so.

RCU's source files have a new home in 3.13, consolidated from the
\co{kernel} directory into a new \co{kernel/rcu} directory.

Finally, Christoph Lameter provided a patch updating RCU's use of
per-CPU-variable APIs and Kirill Tkhai provided a fix for a problem
in which kernels built with \co{CONFIG_RCU_NOCB_CPU_ALL} would
panic on boot when running
on systems with sparse CPU numbering.

\subsection{3.14 Linux Kernel}

The main addition in 3.14 was improvements to the in-kernel rcutorture
test scripts, including a long-overdue refactoring of the test cases.
This release also eliminated a source of OS jitter that was caused
by RCU needlessly undertaking core processing on \co{NO_HZ_FULL} CPUs.
This release also saw a number of fixes, including fixes to Coccinelle
warnings from Fengguang Wu, a first step towards eliminating an
\co{rcu_read_unlock_special()} check by Lai Jiangshan, some buffer-overflow
avoidance from Chen Gang, removal of unnecessary \co{extern} tags by
Teodora Baluta, and improved \co{rcu_assign_pointer()} logic from
Josh Triplett.
